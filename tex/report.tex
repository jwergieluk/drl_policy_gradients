\documentclass[a4paper,12pt]{amsart}
\usepackage[utf8]{inputenc}\usepackage[T1]{fontenc}

\title{Policy Gradient Algorithms}
\author{Julian Wergieluk}\address{}\email{julian.wergieluk@risklab.com}
%%% A generic preamble. 

\usepackage[english]{babel} 


\usepackage[final]{graphicx} % options = [final]  = all graphics displayed, regardless of draft option in class
                               % options = [pdftex] = necessary (?) if import PDF files
                               % no option : when importing ps- and eps-files (?)
% \graphicspath{{../images/}}  % tell LaTeX where to look for images
% \DeclareGraphicsExtensions{.pdf, .PDF, .jpg, .JPG, .jpeg, .JPEG, .png, .PNG, .bmp, .BMP, .eps, .ps}
\usepackage{float}                      % Improved interface for floating objects ; add [H] option

\usepackage[usenames,dvipsnames,svgnames,table]{xcolor}
%\usepackage{listings}%\lstset{basicstyle=\ttfamily, breaklines=true}
\usepackage{verbatim}
%\makeindex

% memoir options
%\chapterstyle{companion}

% ams packages and symbols
\usepackage{amsmath}      % loads amstext, amsbsy, amsopn but not amssymb
                            % equation stuff (eqref, subequations, equation, align, gather, flalign, multline, alignat, split...)
\usepackage{amsfonts}     % may be redundant with amsmath
% \usepackage{amssymb}      % may be redundant with amsmath
\usepackage{latexsym}

% layout
\usepackage[final, babel]{microtype} % many good lay-out/justification effects, see:
                                     % texblog.net/latex-archive/layout/pdflatex-microtype/

\usepackage{mathtools}%                  http://www.ctan.org/pkg/mathtools
\DeclarePairedDelimiter\abs{\lvert}{\rvert}

\usepackage[tableposition=top]{caption}% http://www.ctan.org/pkg/caption
\captionsetup[table]{skip=10pt}
%\usepackage{booktabs,dcolumn}%           http://www.ctan.org/pkg/dcolumn + http://www.ctan.org/pkg/booktabs
%\usepackage{textpos}     % insert a text at an absolute position

\usepackage{amsthm, thmtools}

\theoremstyle{plain}
\newtheorem{theorem}{Theorem}[section]
\newtheorem{acknowledgement}[theorem]{Acknowledgement}
\newtheorem{algorithm}[theorem]{Algorithm}
\newtheorem{axiom}[theorem]{Axiom}
\newtheorem{case}[theorem]{Case}
\newtheorem{claim}[theorem]{Claim}
\newtheorem{conclusion}[theorem]{Conclusion}
\newtheorem{conjecture}[theorem]{Conjecture}
\newtheorem{corollary}[theorem]{Corollary}
\newtheorem{criterion}[theorem]{Criterion}
\newtheorem{exercise}[theorem]{Exercise}
\newtheorem{lemma}[theorem]{Lemma}
\newtheorem{notation}[theorem]{Notation}
\newtheorem{problem}[theorem]{Problem}
\newtheorem{proposition}[theorem]{Proposition}
\newtheorem{solution}[theorem]{Solution}
\newtheorem{definition}[theorem]{Definition}
\newtheorem{condition}[theorem]{Condition}
\newtheorem{assumption}[theorem]{Assumption}
\newtheorem{model}[theorem]{Model}

\theoremstyle{remark}
\newtheorem{remark}[theorem]{Remark}
%\newtheorem{example}[theorem]{Example}
\newtheorem{summary}[theorem]{Summary}
\newtheorem{questions}[theorem]{Questions}

%%% FONTS

%\usepackage{concmath}
\usepackage{fourier}
\usepackage{newcent}


% HYPERREF (last) then HYPCAP %%%%%%%%%%%%%%%%%%%%%%%%%%%%%%%%%%%%%%%%%%%%%%%%%

%
% See:
% http://tex.stackexchange.com/questions/1863/which-packages-should-be-loaded-after-hyperref-instead-of-before
%
\usepackage[
pdftex, 
final,                      % if you do    want to have clickable-colorful links
pdfstartview = FitV,
linktocpage  = false,       % ToC, LoF, LoT place hyperlink on page number, rather than entry text
breaklinks   = true,        % so long urls are correctly broken across lines
% pagebackref  = false,     % add page number in bibliography and link to position in document where cited
]{hyperref}
\hypersetup{
    colorlinks=true, 
    linkcolor={black},
    citecolor={black},
    urlcolor={black},
    bookmarksopenlevel={2}, 
    bookmarksopen = true
}

% \usepackage{cleveref} % enhance cross-referencing, allow full formatting, commands:
                        % \cref, \Cref, \crefrange, \cref{eq2,eq1,eq3,eq5,thm2,def3}
                        % supposedly better than \autoref as provided by hyperref

% \usepackage[all]{hypcap} % when link to float (using hyperref), link anchors to beginning (instead of below) float




% a bar that is slightly shorter than \overline
\newcommand{\overbar}[1]{\mkern 1.5mu\overline{\mkern-1.5mu#1\mkern-1.5mu}\mkern 1.5mu}

% Thesis specific commands
\newcommand{\orderbooks}{\mathcal O}
\newcommand{\supplyfunctions}{\mathcal S}
\newcommand{\demandfunctions}{\mathcal D}
\newcommand{\ubcS}{\overbar\supplyfunctions}
\newcommand{\ubcD}{\overbar\demandfunctions}

% Annotation
\newcommand{\key}[1]{\emph{#1}\index{#1}}
\newcommand{\BUL}{\noindent\quad\textbullet\quad }

% General
\newcommand{\R}{\mathbb{R}}
\newcommand{\eR}{\overbar{\R}}  % extended reals
\newcommand{\N}{\mathbb{N}}
\newcommand{\ind}{1}
\newcommand{\isec}{\star}

% Topology
\newcommand{\interior}[1]{{#1}^{o}}
\newcommand{\closure}[1]{\bar{#1}}

%Stochastics
\newcommand{\E}{\mathbb{E}}
\newcommand{\cond}{\;\middle\vert\;}
\newcommand{\vb}{\;|\;}
\newcommand{\condE}[2]{\E\left[ #1 \,|\, #2 \right] }
\newcommand{\condP}[2]{\bP\left[ #1 \,|\, #2 \right] }
\newcommand\upmodels{\protect\mathpalette{\protect\independenT}{\perp}}
\def\independenT#1#2{\mathrel{\rlap{$#1#2$}\mkern2mu{#1#2}}}
\DeclareMathOperator*{\esssup}{ess\,sup}
\DeclareMathOperator*{\essinf}{ess\,inf}
\newcommand{\sint}{\bullet} %% operator notation for stochastic integral
\newcommand{\dchar}[1]{\partial{#1}} %% differential characteristics of a semimartingale

% Logic
\newcommand{\setcolon}{:}
\newcommand{\impl}{\Rightarrow}
\DeclareMathOperator{\cov}{cov}
\DeclareMathOperator{\corr}{corr}
\DeclareMathOperator{\var}{var}\DeclareMathOperator{\Var}{Var}

\renewcommand{\div}{\mathop\backslash}

\newcommand{\bA}{\mathbb A}\newcommand{\cA}{\mathcal A}\newcommand{\fA}{\mathfrak A}\newcommand{\bfA}{\mathbf A}
\newcommand{\bB}{\mathbb B}\newcommand{\cB}{\mathcal B}\newcommand{\fB}{\mathfrak B}\newcommand{\bfB}{\mathbf B}
\newcommand{\bC}{\mathbb C}\newcommand{\cC}{\mathcal C}\newcommand{\fC}{\mathfrak C}\newcommand{\bfC}{\mathbf C}
\newcommand{\bD}{\mathbb D}\newcommand{\cD}{\mathcal D}\newcommand{\fD}{\mathfrak D}\newcommand{\bfD}{\mathbf D}
\newcommand{\bE}{\mathbb E}\newcommand{\cE}{\mathcal E}\newcommand{\fE}{\mathfrak E}\newcommand{\bfE}{\mathbf E}
\newcommand{\bF}{\mathbb F}\newcommand{\cF}{\mathcal F}\newcommand{\fF}{\mathfrak F}\newcommand{\bfF}{\mathbf F}
\newcommand{\bG}{\mathbb G}\newcommand{\cG}{\mathcal G}\newcommand{\fG}{\mathfrak G}\newcommand{\bfG}{\mathbf G}
\newcommand{\bH}{\mathbb H}\newcommand{\cH}{\mathcal H}\newcommand{\fH}{\mathfrak H}\newcommand{\bfH}{\mathbf H}
\newcommand{\bI}{\mathbb I}\newcommand{\cI}{\mathcal I}\newcommand{\fI}{\mathfrak I}\newcommand{\bfI}{\mathbf I}
\newcommand{\bJ}{\mathbb J}\newcommand{\cJ}{\mathcal J}\newcommand{\fJ}{\mathfrak J}\newcommand{\bfJ}{\mathbf J}
\newcommand{\bK}{\mathbb K}\newcommand{\cK}{\mathcal K}\newcommand{\fK}{\mathfrak K}\newcommand{\bfK}{\mathbf K}
\newcommand{\bL}{\mathbb L}\newcommand{\cL}{\mathcal L}\newcommand{\fL}{\mathfrak L}\newcommand{\bfL}{\mathbf L}
\newcommand{\bM}{\mathbb M}\newcommand{\cM}{\mathcal M}\newcommand{\fM}{\mathfrak M}\newcommand{\bfM}{\mathbf M}
\newcommand{\bN}{\mathbb N}\newcommand{\cN}{\mathcal N}\newcommand{\fN}{\mathfrak N}\newcommand{\bfN}{\mathbf N}
\newcommand{\bO}{\mathbb O}\newcommand{\cO}{\mathcal O}\newcommand{\fO}{\mathfrak O}\newcommand{\bfO}{\mathbf O}
\newcommand{\bP}{\mathbb P}\newcommand{\cP}{\mathcal P}\newcommand{\fP}{\mathfrak P}\newcommand{\bfP}{\mathbf P}
\newcommand{\bQ}{\mathbb Q}\newcommand{\cQ}{\mathcal Q}\newcommand{\fQ}{\mathfrak Q}\newcommand{\bfQ}{\mathbf Q}
\newcommand{\bR}{\mathbb R}\newcommand{\cR}{\mathcal R}\newcommand{\fR}{\mathfrak R}\newcommand{\bfR}{\mathbf R}
\newcommand{\bS}{\mathbb S}\newcommand{\cS}{\mathcal S}\newcommand{\fS}{\mathfrak S}\newcommand{\bfS}{\mathbf S}
\newcommand{\bT}{\mathbb T}\newcommand{\cT}{\mathcal T}\newcommand{\fT}{\mathfrak T}\newcommand{\bfT}{\mathbf T}
\newcommand{\bU}{\mathbb U}\newcommand{\cU}{\mathcal U}\newcommand{\fU}{\mathfrak U}\newcommand{\bfU}{\mathbf U}
\newcommand{\bW}{\mathbb W}\newcommand{\cW}{\mathcal W}\newcommand{\fW}{\mathfrak W}\newcommand{\bfW}{\mathbf W}
\newcommand{\bV}{\mathbb V}\newcommand{\cV}{\mathcal V}\newcommand{\fV}{\mathfrak V}\newcommand{\bfV}{\mathbf V}
\newcommand{\bX}{\mathbb X}\newcommand{\cX}{\mathcal X}\newcommand{\fX}{\mathfrak X}\newcommand{\bfX}{\mathbf X}
\newcommand{\bY}{\mathbb Y}\newcommand{\cY}{\mathcal Y}\newcommand{\fY}{\mathfrak Y}\newcommand{\bfY}{\mathbf Y}
\newcommand{\bZ}{\mathbb Z}\newcommand{\cZ}{\mathcal Z}\newcommand{\fZ}{\mathfrak Z}\newcommand{\bfZ}{\mathbf Z}

%% chap. 1

\newcommand{\ciso}{\tilde s}
\newcommand{\liso}{\tilde s}
\newcommand{\iso}{operator}



\usepackage[url=false,backend=biber]{biblatex}
\addbibresource{literature.bib}

\newcommand{\stateSpace}{\mathbb S}
\newcommand{\stateSpaceAlg}{\mathcal S}
\newcommand{\actionSpace}{\mathbb A}
\newcommand{\actionSpaceAlg}{\mathcal A}
\newcommand{\policy}{\pi}
\newcommand{\discountFactor}{\gamma}
\newcommand{\prob}{\mathbb P}
\newcommand{\rewardFunc}{\phi}
\newcommand{\trajectory}{\tau}
\newcommand{\trajectorySpace}{\mathbb T}
\newcommand{\startStateDist}{\rho_0}

\begin{document}

\maketitle

\begin{abstract}
This short note provides a concise description of the model architecture and
learning algorithms of the agent developed in this project. We also report learning
performance of the agent and provide a list of possible future model improvements.
\end{abstract}
\renewcommand*{\thefootnote}{}\footnote{\today{}}

\section{Description of the learning algorithm}

%Requirement: The report clearly describes the learning algorithm, along with
%the chosen hyperparameters. It also describes the model architectures for any
%neural networks.

\subsection{Summary of the notation}

In this write-up I am going to cast the core ideas of policy gradient methods
into a rigorous probabilistic framework. Much of the confusion and
difficulty of understanding reinforcement learning comes from sloppy and
incomplete mathematical notation.

Let's setup up the stage for the show and define a probability space
$\left( \Omega, \mathcal F, \prob \right)$.
Consider a Markov Decision Process (MDP) with measurable state and action spaces
$(\stateSpace, \stateSpaceAlg)$ and $(\actionSpace, \actionSpaceAlg)$.
A policy $\policy$ is a Markov kernel from $(\stateSpace, \stateSpaceAlg)$ to
$(\actionSpace, \actionSpaceAlg)$, i.e.\ for each $s\in\stateSpace$, 
$\policy(. \vb s)$ is a probability distribution on $\actionSpaceAlg$.
A trajectory
\begin{align*}
    \trajectory = \left( S_0, A_0, R_1, S_1, A_1, R_2, \ldots \right) 
\end{align*}
encodes a sequence of states, actions and resulting numerical rewards
pertaining to a policy $\policy$ indexed by $t\in\N$. The elements of $\tau$
are random variables, such that $S_{0}$ has the \key{start-state distribution} 
$\startStateDist$ (which does not depend on the policy $\pi$), 
$\policy(. | s)$ is the conditional distribution of $A_i$ given $S_i = s$ for
any $s\in\stateSpace$, and $\prob\left(. | S_{i-1}= s, A_{i-1}=a \right)$ is
the conditional distribution of $S_{i}$.  The rewards $R_i$ are random
variables taking values in $\R$.  The return or discounted future reward $G_t$
at $t\in\N$ is defined as
\begin{align*}
    G_t &= \sum_{k=0}^{\infty} \discountFactor^{k} R_{t+k+1}.
\end{align*}
Since $\trajectory$ is a sequence of random variables, and therefore itself a random
variable taking values in the trajectory space 
\begin{align*}
    \trajectorySpace = \stateSpace\times\actionSpace\times 
        \left( \stateSpace\times\actionSpace\times\R \right)^{\infty},
\end{align*}
we can sample from $\trajectory$ to obtain sequences of the form
\begin{align*}
    \left( s_0, a_0, r_1, s_1, a_1, r_2,\ldots \right).
\end{align*}
These realizations of $\tau$ are called \key{episodes} or \key{rollouts}.

The central object of interest is the mechanics of the world encoded in the
transition probability measure $\prob(. \vb s, a)$. For $i>0$, it is the distribution of
$S_i$ under the conditions $S_{i-1} = s$ and $A_{i-1} = a$ for fixed
$s\in\stateSpace$ and $a\in\actionSpace$. This distribution does not depend on $i$. 

To simplify the setup, we assume that the reward $R_i$ at time $i$ is a deterministic
function of the relevant states and actions. Depending on the problem at hand
these are the three common choices encountered in the literature:
\begin{align}
    R_{i} &= \rewardFunc_{3}(S_{i-1}, A_{i-1}, S_{i}) \\
    R_{i} &= \rewardFunc_{2}(A_{i-1}, S_{i}) \\
    R_{i} &= \rewardFunc_{1}(S_{i})
\end{align}

In reinforcement learning, we parametrize the policy $\policy_\theta$ with
an $\R^{d}$ vector $\theta$. The goal is to find good values for $\theta$ 
which lead to high cumulative expected reward
\begin{align*}
    J\left( \theta \right) &= \E_{\theta} \left[ G_0 \right] 
    = \E_{\theta} \left[ \sum_{i=0}^{\infty} R_{i+1} \right].
\end{align*}
The subscript $\theta$ next to the expectation operator indicates that
the underlying probability measure $\prob_{\theta}$ combines both
$\prob$ and $\policy_{\theta}$.


\section{Training analysis}

\section{Ideas for future work}

%\nocite{meucci2009review, goodman2007}
\printbibliography

\end{document}


\begin{figure}[tb]
    \centering
    \includegraphics[width=\textwidth]{{epsilon}.pdf}
    \caption{Decay of the $\varepsilon$ parameter as function the 
    training episode number.}
    \label{fig:epsilon}
\end{figure}

% vim: spelllang=en_us:spell:
